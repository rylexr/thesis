\batchmode


\documentclass[12pt,letterpaper,twoside,openright]{book}
\RequirePackage{ifthen}




\usepackage[spanish,english]{babel}


\usepackage{mathpazo} 
\linespread{1.05}        % Palatino needs more leading (space between lines)
\usepackage[scaled]{helvet} 
\usepackage{courier} 
\normalfont
\usepackage[T1]{fontenc}


\usepackage{subfig}


\usepackage{indentfirst}


\raggedbottom


\usepackage[hmargin={3.5cm,2.5cm},vmargin=2.5cm]{geometry}


\usepackage{natbib}
\bibpunct{[}{]}{,}{n}{,}{,}


\usepackage{setspace}


\usepackage[pdftex]{graphicx}


\usepackage[utf8x]{inputenc}


\usepackage[Sonny]{fncychap} 
\ChNameUpperCase
\ChNameVar{\raggedleft\huge\rm }
\ChRuleWidth{1pt}
\ChTitleUpperCase
\ChNumVar{\raggedleft\bfseries\huge }
\ChTitleVar{\raggedleft\Large\rm }


\setcounter{secnumdepth}{4}
\setcounter{tocdepth}{4}


\usepackage{fancyhdr}


\usepackage{mdwlist}
\usepackage{alltt}
\usepackage[dvips, bookmarks=true,linktoc=all, colorlinks=false, pdftitle={Modelado r\'{a}pido de objetos tridimensionales}, pdfauthor={Randy Saborio}, pdfsubject={Modelado r\'{a}pido de objetos tridimensionales}, pdfkeywords={Tesis, Visión Artificial, 3D, Linux, OpenCV, Reconstrucción 3D},linkcolor=blue,filecolor=blue,urlcolor=blue,citecolor=blue]{hyperref}


\usepackage{epigraph}



%
\newenvironment{abstract}%
{\cleardoublepage\null \vfill \begin{center}%
\bfseries \abstractname \end{center}}
{\vfill\null} 


\usepackage{nomencl}
\makenomenclature


\usepackage{acronym}


\usepackage{color}

%
\providecommand{\HRule}{\rule{\linewidth}{0.5mm}} 





\pagecolor[gray]{.7}

\usepackage[latin1]{inputenc}



\makeatletter
\AtBeginDocument{\makeatletter
\input /home/rylexr/Documents/ITCR/Tesis/LateX/Proposal/main.aux
\makeatother
}

\makeatletter
\count@=\the\catcode`\_ \catcode`\_=8 
\newenvironment{tex2html_wrap}{}{}%
\catcode`\<=12\catcode`\_=\count@
\newcommand{\providedcommand}[1]{\expandafter\providecommand\csname #1\endcsname}%
\newcommand{\renewedcommand}[1]{\expandafter\providecommand\csname #1\endcsname{}%
  \expandafter\renewcommand\csname #1\endcsname}%
\newcommand{\newedenvironment}[1]{\newenvironment{#1}{}{}\renewenvironment{#1}}%
\let\newedcommand\renewedcommand
\let\renewedenvironment\newedenvironment
\makeatother
\let\mathon=$
\let\mathoff=$
\ifx\AtBeginDocument\undefined \newcommand{\AtBeginDocument}[1]{}\fi
\newbox\sizebox
\setlength{\hoffset}{0pt}\setlength{\voffset}{0pt}
\addtolength{\textheight}{\footskip}\setlength{\footskip}{0pt}
\addtolength{\textheight}{\topmargin}\setlength{\topmargin}{0pt}
\addtolength{\textheight}{\headheight}\setlength{\headheight}{0pt}
\addtolength{\textheight}{\headsep}\setlength{\headsep}{0pt}
\setlength{\textwidth}{349pt}
\newwrite\lthtmlwrite
\makeatletter
\let\realnormalsize=\normalsize
\global\topskip=2sp
\def\preveqno{}\let\real@float=\@float \let\realend@float=\end@float
\def\@float{\let\@savefreelist\@freelist\real@float}
\def\liih@math{\ifmmode$\else\bad@math\fi}
\def\end@float{\realend@float\global\let\@freelist\@savefreelist}
\let\real@dbflt=\@dbflt \let\end@dblfloat=\end@float
\let\@largefloatcheck=\relax
\let\if@boxedmulticols=\iftrue
\def\@dbflt{\let\@savefreelist\@freelist\real@dbflt}
\def\adjustnormalsize{\def\normalsize{\mathsurround=0pt \realnormalsize
 \parindent=0pt\abovedisplayskip=0pt\belowdisplayskip=0pt}%
 \def\phantompar{\csname par\endcsname}\normalsize}%
\def\lthtmltypeout#1{{\let\protect\string \immediate\write\lthtmlwrite{#1}}}%
\newcommand\lthtmlhboxmathA{\adjustnormalsize\setbox\sizebox=\hbox\bgroup\kern.05em }%
\newcommand\lthtmlhboxmathB{\adjustnormalsize\setbox\sizebox=\hbox to\hsize\bgroup\hfill }%
\newcommand\lthtmlvboxmathA{\adjustnormalsize\setbox\sizebox=\vbox\bgroup %
 \let\ifinner=\iffalse \let\)\liih@math }%
\newcommand\lthtmlboxmathZ{\@next\next\@currlist{}{\def\next{\voidb@x}}%
 \expandafter\box\next\egroup}%
\newcommand\lthtmlmathtype[1]{\gdef\lthtmlmathenv{#1}}%
\newcommand\lthtmllogmath{\dimen0\ht\sizebox \advance\dimen0\dp\sizebox
  \ifdim\dimen0>.95\vsize
   \lthtmltypeout{%
*** image for \lthtmlmathenv\space is too tall at \the\dimen0, reducing to .95 vsize ***}%
   \ht\sizebox.95\vsize \dp\sizebox\z@ \fi
  \lthtmltypeout{l2hSize %
:\lthtmlmathenv:\the\ht\sizebox::\the\dp\sizebox::\the\wd\sizebox.\preveqno}}%
\newcommand\lthtmlfigureA[1]{\let\@savefreelist\@freelist
       \lthtmlmathtype{#1}\lthtmlvboxmathA}%
\newcommand\lthtmlpictureA{\bgroup\catcode`\_=8 \lthtmlpictureB}%
\newcommand\lthtmlpictureB[1]{\lthtmlmathtype{#1}\egroup
       \let\@savefreelist\@freelist \lthtmlhboxmathB}%
\newcommand\lthtmlpictureZ[1]{\hfill\lthtmlfigureZ}%
\newcommand\lthtmlfigureZ{\lthtmlboxmathZ\lthtmllogmath\copy\sizebox
       \global\let\@freelist\@savefreelist}%
\newcommand\lthtmldisplayA{\bgroup\catcode`\_=8 \lthtmldisplayAi}%
\newcommand\lthtmldisplayAi[1]{\lthtmlmathtype{#1}\egroup\lthtmlvboxmathA}%
\newcommand\lthtmldisplayB[1]{\edef\preveqno{(\theequation)}%
  \lthtmldisplayA{#1}\let\@eqnnum\relax}%
\newcommand\lthtmldisplayZ{\lthtmlboxmathZ\lthtmllogmath\lthtmlsetmath}%
\newcommand\lthtmlinlinemathA{\bgroup\catcode`\_=8 \lthtmlinlinemathB}
\newcommand\lthtmlinlinemathB[1]{\lthtmlmathtype{#1}\egroup\lthtmlhboxmathA
  \vrule height1.5ex width0pt }%
\newcommand\lthtmlinlineA{\bgroup\catcode`\_=8 \lthtmlinlineB}%
\newcommand\lthtmlinlineB[1]{\lthtmlmathtype{#1}\egroup\lthtmlhboxmathA}%
\newcommand\lthtmlinlineZ{\egroup\expandafter\ifdim\dp\sizebox>0pt %
  \expandafter\centerinlinemath\fi\lthtmllogmath\lthtmlsetinline}
\newcommand\lthtmlinlinemathZ{\egroup\expandafter\ifdim\dp\sizebox>0pt %
  \expandafter\centerinlinemath\fi\lthtmllogmath\lthtmlsetmath}
\newcommand\lthtmlindisplaymathZ{\egroup %
  \centerinlinemath\lthtmllogmath\lthtmlsetmath}
\def\lthtmlsetinline{\hbox{\vrule width.1em \vtop{\vbox{%
  \kern.1em\copy\sizebox}\ifdim\dp\sizebox>0pt\kern.1em\else\kern.3pt\fi
  \ifdim\hsize>\wd\sizebox \hrule depth1pt\fi}}}
\def\lthtmlsetmath{\hbox{\vrule width.1em\kern-.05em\vtop{\vbox{%
  \kern.1em\kern0.8 pt\hbox{\hglue.17em\copy\sizebox\hglue0.8 pt}}\kern.3pt%
  \ifdim\dp\sizebox>0pt\kern.1em\fi \kern0.8 pt%
  \ifdim\hsize>\wd\sizebox \hrule depth1pt\fi}}}
\def\centerinlinemath{%
  \dimen1=\ifdim\ht\sizebox<\dp\sizebox \dp\sizebox\else\ht\sizebox\fi
  \advance\dimen1by.5pt \vrule width0pt height\dimen1 depth\dimen1 
 \dp\sizebox=\dimen1\ht\sizebox=\dimen1\relax}

\def\lthtmlcheckvsize{\ifdim\ht\sizebox<\vsize 
  \ifdim\wd\sizebox<\hsize\expandafter\hfill\fi \expandafter\vfill
  \else\expandafter\vss\fi}%
\providecommand{\selectlanguage}[1]{}%
\makeatletter \tracingstats = 1 


\begin{document}
\pagestyle{empty}\thispagestyle{empty}\lthtmltypeout{}%
\lthtmltypeout{latex2htmlLength hsize=\the\hsize}\lthtmltypeout{}%
\lthtmltypeout{latex2htmlLength vsize=\the\vsize}\lthtmltypeout{}%
\lthtmltypeout{latex2htmlLength hoffset=\the\hoffset}\lthtmltypeout{}%
\lthtmltypeout{latex2htmlLength voffset=\the\voffset}\lthtmltypeout{}%
\lthtmltypeout{latex2htmlLength topmargin=\the\topmargin}\lthtmltypeout{}%
\lthtmltypeout{latex2htmlLength topskip=\the\topskip}\lthtmltypeout{}%
\lthtmltypeout{latex2htmlLength headheight=\the\headheight}\lthtmltypeout{}%
\lthtmltypeout{latex2htmlLength headsep=\the\headsep}\lthtmltypeout{}%
\lthtmltypeout{latex2htmlLength parskip=\the\parskip}\lthtmltypeout{}%
\lthtmltypeout{latex2htmlLength oddsidemargin=\the\oddsidemargin}\lthtmltypeout{}%
\makeatletter
\if@twoside\lthtmltypeout{latex2htmlLength evensidemargin=\the\evensidemargin}%
\else\lthtmltypeout{latex2htmlLength evensidemargin=\the\oddsidemargin}\fi%
\lthtmltypeout{}%
\makeatother
\setcounter{page}{1}
\onecolumn

% !!! IMAGES START HERE !!!

{\newpage\clearpage
\lthtmlinlineA{tex2html_nomath_inline1909}%
\textquestiondown%
\lthtmlinlineZ
\lthtmlcheckvsize\clearpage}

\setcounter{secnumdepth}{4}
\setcounter{tocdepth}{4}
{\newpage\clearpage
\lthtmlpictureA{tex2html_wrap2490}%
\includegraphics[width=0.3\textwidth]{images/logotec}%
\lthtmlpictureZ
\lthtmlcheckvsize\clearpage}



\selectlanguage{spanish}



\selectlanguage{spanish}

\stepcounter{chapter}
\stepcounter{section}
\stepcounter{section}
{\newpage\clearpage
\lthtmlfigureA{itemizestar322}%
\begin{itemize*}
\item La detecci\'{o}n, segmentaci\'{o}n, localizaci\'{o}n y reconocimiento de ciertos objetos en im\'{a}genes (por ejemplo, caras humanas). 
\item La evaluaci\'{o}n de los resultados (por ejemplo, segmentaci\'{o}n, registro). 
\item Registro de diferentes im\'{a}genes de una misma escena u objeto, es decir, hacer concordar un mismo objeto en diversas im\'{a}genes. 
\item Seguimiento de un objeto en una secuencia de im\'{a}genes. 
\item Mapeo de una escena para generar un modelo tridimensional; este modelo podr\'{i}a ser usado por un robot para navegar por la escena. 
\item Estimaci\'{o}n de las posturas tridimensionales de humanos. 
\item B\'{u}squeda de im\'{a}genes digitales por su contenido. 
\end{itemize*}%
\lthtmlfigureZ
\lthtmlcheckvsize\clearpage}

\stepcounter{subsection}
\stepcounter{subsection}
\stepcounter{subsubsection}
\stepcounter{subsubsection}
\stepcounter{subsubsection}
\stepcounter{section}
\stepcounter{section}
\stepcounter{section}
\stepcounter{chapter}
\stepcounter{section}
\stepcounter{section}
{\newpage\clearpage
\lthtmlfigureA{itemizestar821}%
\begin{itemize*}
\item Desarrollar algoritmos para reconstruir el objeto 3D conforme la informaci\'{o}n de \'{e}ste es capturada.
\item Establecer la relaci\'{o}n entre resoluci\'{o}n de im\'{a}genes y tiempo de reconstrucci\'{o}n 3D.
\item Establecer la relación entre caracter\'{i}sticas f\'{i}sicas (tamaño, forma, color, textura) del objeto y tiempo de reconstrucci\'{o}n 3D.
\item Establecer la relaci\'{o}n entre calibraci\'{o}n y calidad de la reconstrucci\'{o}n 3D.
\item Analizar y valorar t\'{e}cnicas actuales de reconstrucci\'{o}n 3D.
\item Mantener considerablemente bajo el costo global del sistema.
\end{itemize*}%
\lthtmlfigureZ
\lthtmlcheckvsize\clearpage}

\stepcounter{section}
\stepcounter{section}
{\newpage\clearpage
\lthtmlfigureA{itemizestar865}%
\begin{itemize*}
\item Reconstrucci\'{o}n de objetos no r\'{i}gidos.
\item Reconstrucci\'{o}n de objetos con un ancho, alto y profundidad mayor a 30cm.
\item Reconstrucci\'{o}n de objetos con texturas muy uniformes.
\item Reconstrucci\'{o}n de objetos con estructuras muy irregulares.
\item Etapas posteriores a la triangulaci\'{o}n (\emph{bundle adjusment}, \emph{texturing}).
\item Utilizaci\'{o}n de equipo especializado (c\'{a}maras infrarrojas, de profundidad, proyectores, entre otros) o computadores de alto rendimiento.
\end{itemize*}%
\lthtmlfigureZ
\lthtmlcheckvsize\clearpage}

\stepcounter{chapter}
\stepcounter{section}
\stepcounter{section}
\stepcounter{section}
{\newpage\clearpage
\lthtmlfigureA{enumeratestar935}%
\begin{enumerate*}
\item Prototipo funcional del sistema de reconstrucci\'{o}n de objetos tridimensionales.
\item Una video-presentaci\'{o}n mostrando la reconstrucci\'{o}n de objetos tri\-di\-men\-sio\-na\-les utilizando el sistema.
\end{enumerate*}%
\lthtmlfigureZ
\lthtmlcheckvsize\clearpage}

\stepcounter{section}
\refstepcounter{chapter}
\stepcounter{chapter}
{\newpage\clearpage
\lthtmlfigureA{acronym1001}%
\begin{acronym}
\par
\acro{2D}{Bidimensional} es que posee dos dimensiones.
\par
\acro{3D}{Tridimensional} es que posee tres dimensiones.
\par
\acro{CCD}{Charge-Coupled Device} es un circuito integrado que contiene un número determinado de condensadores enlazados o acoplados. Bajo el control de un circuito interno, cada condensador puede transferir su carga eléctrica a uno o a varios de los condensadores que estén a su lado en el circuito impreso.
\par
\acro{SLAM}{Simultaneous Localization And Mapping} es una t\'{e}cnica usada por robots y veh\'{i}culos aut\'{o}nomos para construir un mapa de un entorno desconocido en el que se encuentra, a la vez que estima su trayectoria al desplazarse dentro de este entorno.
\par
\end{acronym}%
\lthtmlfigureZ
\lthtmlcheckvsize\clearpage}


\end{document}
