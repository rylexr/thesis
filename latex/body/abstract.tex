\setstretch{1.55}

\selectlanguage{english}%
\begin{abstract}
The demand for three-dimensional (3D) virtual models of complex objects has been increasing in many fields. Highly realistic 3D models are daily used to visualize and simulate events in the medical field, games, architecture, film industry, 3D printing, among others. One of the major limitations to the widespread use of these techniques is the high cost of these 3D models as they are elaborated manually. If the object to reconstruct comes from the real world, the process involves planning a set of views, physically altering and manipulating the position of the object and the sensor, capturing the object, recording the acquired geometric data into a common framework, and finally, integrating the acquired information into a virtual, consistent, and non-redundant model. Given the nature of this process, validating the correctness of the generated model involves much time, effort, and money.

This paper presents a fast three-dimensional reconstruction technique using only pairs of stereoscopic views of the object to reconstruct taken with a low-cost camera (webcam). The purpose of this technique is to reduce the cost of the reconstruction process by not requiring specialized equipment or personnel, as well as reducing its duration. The novelty of this technique is that it uses dense pixel tracking algorithms along with a pyramid technique for fast image processing and a rapid triangulation process to generate three-dimensional information. Generally, dense pixel tracking is not suitable for use in rapid reconstruction due to the considerable amount of data to be processed, however the proposed technique allows to obtain high quality results within a few seconds without using any specialized equipment.

\textbf{Keywords:} computer vision, fast 3D dense reconstruction.
\end{abstract}

\selectlanguage{spanish}%
\begin{abstract}
La demanda por modelos tridimensionales (3D) virtuales de objetos f\'{i}sicos complejos ha venido en crecimiento en una gran cantidad de \'{a}reas. Modelos 3D de gran realismo son utilizados a diario para visualizar y simular eventos en el campo m\'{e}dico, en juegos, arquitectura, en la industria f\'{i}lmica, en impresi\'{o}n 3D, entre otros. Una de las grandes limitantes para el uso generalizado de este tipo de t\'{e}cnicas es el costo elevado de estos modelos 3D dado que son producidos manualmente. Si los objetos a reconstruir provienen del mundo real, el proceso implica planificar un conjunto de vistas, alterar y manipular f\'{i}sicamente la posici\'{o}n del objeto y del sensor, realizar capturas del objeto, registrar los datos geom\'{e}tricos adquiridos en un marco de referencia com\'{u}n, y finalmente, integrar la informaci\'{o}n adquirida en un modelo virtual consistente y no redundante. Dada la naturaleza de este proceso, validar la correctitud del modelo generado implica mucho tiempo, esfuerzo y dinero.

El presente trabajo muestra una t\'{e}cnica r\'{a}pida de reconstrucci\'{o}n tridimensional, utilizando \'{u}nicamente pares de vistas estereosc\'{o}picas del objeto a reconstruir tomadas con una c\'{a}mara de bajo costo tipo \textit{webcam}. Con dicha t\'{e}cnica se pretende reducir los costos del proceso de reconstrucci\'{o}n, al no requerir de equipo ni personal especializado, as\'{i} como su duraci\'{o}n. La novedad de la t\'{e}cnica es que utiliza algoritmos de rastreo denso de pixeles en conjunto con una t\'{e}cnica pir\'{a}mide para el r\'{a}pido procesamiento de las im\'{a}genes y una r\'{a}pida triangulaci\'{o}n para la generaci\'{o}n de la informaci\'{o}n tridimensional. Generalmente, el rastreo denso no es apto para ser utilizado en reconstrucci\'{o}n r\'{a}pida debido a la cantidad considerable de datos que hay que procesar, sin embargo la t\'{e}cnica propuesta permite obtener resultados de gran calidad en cuesti\'{o}n de unos cuantos segundos y sin la utilizaci\'{o}n de equipo especializado.

\textbf{Palabras clave:} visi\'{o}n artificial, r\'{a}pida reconstrucci\'{o}n densa 3D.
\end{abstract}

\singlespace